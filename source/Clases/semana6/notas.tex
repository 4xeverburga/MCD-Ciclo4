\section*{Semana5}
\justifying

\subsection*{Primeras 2 horas}

\subsubsection*{Eliticación o levantamiento de requerimientos}

Ingeniería de software Ian Sommervile

% make a list
\begin{itemize}
\item Entrevistas
\item Encuestas/cuestionarios
\item Documentación
\item Historias de usuario
\item Tormenta de ideas
\item Observación
\item Workshops
\end{itemize}

\paragraph*{Atributo} Dato que contiene descripción parcial de un ente mayor. Pueden ser estáticos o dinámicos. Los datos estáticos pueden ser parque o parámetro. 

\paragraph*{Atributo parámetro} Permiten configurara características tipo catálogo. Se describen a los objetos reales o posibles
. La importancia de las tipologías es que sirven de patrón de los objetos de la realidad para calsificarlos, afectarlos y procesarlos.

La consignación de producto es acordar una comisión para cuando se efectúe la venta de un producto. 
Matriz de evaluación de riesgos para la gestión de personal. Es importante que las personas 
encargadas de rellenar los datos lo hagan a tiempo. De igual modo que se introduzcan los datos de manera correcta 
al sistema de información: síndrome del dedo gordo. 

Un principio del desarrollo de sistemas es: "a prueba de imbéciles". 


\paragraph*{Opreación de Negocio} Es el objeto de transacción creada. Indica los resultados en función de los datos del parque de entrada.

Entrada -> Proceso -> Resultados

Mucha data dinámica es de tipo cuenta, lo que quiere decir que puede ser contabilizada. 



\subsubsection*{Naturaleza de comportamientos} 

Los atributos son generados por los eventos consumados. 

ES IMPORTANTE QUE LA SEMÁNTICA ESTÉ BIEN DEFINIDA EN EL DICCIONARIO SEMÁNTICO Y ONTOLÓGICO DE DATOS


Se puede utilizar un dígito para detección de errores en una secuencia. 