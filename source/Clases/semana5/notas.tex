\section*{Semana5}
\justifying

\subsection*{Primeras 2 horas}

Tendencias socio-económicas. Visionado.

Todo gerente tiene que ver el futuro. No se puede asumir que todo va a seguir igual.
El diseñador de la base de datos tiene que prepararse para los posibles cambios que el futuro pueda tener.

El analista de datos debe conocer cómo funciona toda la empresa. 

misión --> visión --> funciones --> eventos --> operaciones

En cambio, en el futuro:

Rhuanda (economía)
% make a list
\begin{itemize}
\item Visión puede cambiar, funciones pueden cambiar y afectan a los eventos y operaciones
\item Segmentación del individuo: productos personalizados para cada personalizados
\item Cambios demográficos: la población con mayores ingresos son la gente mayor. El machismo genera diferencias de compras por género e acuerdo a los estereotipos que se crean
\item Sistema multidivisa: datos para distintas divisas. Cutoas en divisa extranjera. Se necesitan dierentes tablas.
\item Achicar/Diversificar la cadena de suministros: tener varios proveedores en cierta proporciones. Cuando ocurra un desastre se tiene más de un proveedor al que acudir.
\item Precarización del empleo / desempleo al ocurrir automatización. Aumento de contratos precarios
\item Inclusividad: mascostas, géneros no binarios, mujeres, etc.
\end{itemize}

Funciones en la uni:

\paragraph*{Ventas} Se realiza publicidad, relaciones públicas. Admisión del cliente (Ejemplo: una empresa de seguros). Matrícula. Seguimiento del egresado
\paragraph*{Logística} Eventos: empieza con la matrícula en el curso, 
dictado de clases, evaluación, entrega de notas.
Las operaciones del evento dictado de clases    
\paragraph*{RR.Humanos} 
\paragraph*{Pedagógica}
\paragraph*{Investigación}

\subsection*{Últimas 2 horas}
La manera más natural de los humanos modernos para pensar es con el lenguaje. 
No hay conceptos autocontenidos. No existen palabras aisladas. 



\subsubsection*{Marco Semántico-Ontológico}
\paragraph*{Semántica} En base a significado de las palabras. Lo que el diccionario diría: se necesita usar las palabras correctas para modelar.
Es la primera aproximación al modelado.  
\paragraph*{Ontología} Es el estudio de las cosas estudiando sus atributos. Se toma la data
y se empieza a agrupar a los datos en entidades. Muchas veces las entidades se pueden corresponder a clases en 
el modelo lógico.
 
\paragraph*{Taxonomía} La clasificación

¿El ser humano puede razonar sin lenguaje? El idioma o lenguaje es también un modelo; aproxima la realidad, no es perfecta.
La forma en la que un modelo se puede expresar de manera lógica es con un LENGUAJE de modelado. 

La técnica semántica se enfoca en definir la palabra para acercarla lo más posible al objeto real con el símbolo lingüístico. 

% make a note about Sartori 
Australasian Journal of information systems.
Sartori definition concepts. Concepts and method in social science: The tradition of Giovanni Sartori
\paragraph*{El trabajo de Sartori} 
\begin{itemize}
    \item La definición se hace en comunidad
    \item La definición es un paraguas. Debajo el paraguas están todos los que son y deben estar.  
    \item Algunos conceptos se deben definir por exclusión o por aglomeración. \\
    entonces se puede definir dictadura como los gobiernos que no son democracia. 
    Por aglomeración se pueden tomar varios conceptos pequeños para crear uno más grande. 
\end{itemize}

\paragraph*{Ejemplo} Definir lo que es un cliente. 


Si un cliente fuese el que se beneficia de un servicio, entonces los padres 
de los alummnos de una universidad no pueden ser clientes. Si fuese el que paga,
entonces los alumnos se quedan fuera. 

Una mejor definición de cliente puede ser: Cualquier persona (humana, animal o jurídica) que adquiere algún
derecho sobre los productos/servicios de la empresa. 

Siendo así, cuando quiebre un colegio, entonces le debe el derecho de los
padres o a los que hatan pagado la pensión.
\paragraph*{Ontología} 
\begin{itemize}
    \item Representación del conocimiento mediante conceptos de alto poder expresibo. 
    \item  
\end{itemize}
\subsubsection*{Principios del modelado}

\begin{itemize}
    \item Objetivos: solo usan datos. No se pueden colocar datos con mucha senstitividad. 
    \item Semántica: significado del dato. Incluye métodos y a naturaleza de los datos.
    \item Propiedad: Los procesos y los datos se modelan en paralelo
    \item Técnica: Se tendrán que agrupar valores en un categoría llamada entidad.
    Nosotros partimos de los atributos hacia las entidades. La semántica nos ayuda a discernir
    qué datos en realidad van en. Por ejemplo, un celular en específico 
    no tiene fecha de vencimiento; la fecha de vencimiento es del item. 
    
\end{itemize}

Se investiga la naturaleza de los datos. Las definiciones dan lugar a 
definir las características de los datos. Si hay un dato que no es tocado por ningún metodo
es porque debe estar para utilizarse para un evento futuro. Si no, es innecesario. 

En un stockitem puede incluirse información sobre cierto lote
fecha\_lote, fecha\_vencimiento, etc. 
Más info en las diapos.

\subsubsection*{Atributos estáticos-dinámicos}